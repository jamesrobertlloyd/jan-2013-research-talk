\input{header_beamer}

\usecolortheme{default}
\xdefinecolor{Black}{rgb}{0,0,0}
\xdefinecolor{White}{rgb}{1,1,1}
\xdefinecolor{DarkBlue}{rgb}{0,0,.7}
\xdefinecolor{DarkRed}{rgb}{.7,0,0}
\xdefinecolor{Red}{rgb}{.85,0,0}
\xdefinecolor{DarkGreen}{rgb}{0,.7,0}
\xdefinecolor{DarkMagenta}{rgb}{.6,0,.6}
\def\Black{\textcolor{Black}}
\def\White{\textcolor{White}}
\def\Blue{\textcolor{DarkBlue}}
\def\Magenta{\textcolor{DarkMagenta}}
\def\Red{\textcolor{Red}}
\def\Green{\textcolor{DarkGreen}}

\usepackage{alltt}
\usepackage{psfrag}
\usepackage{pstool}

\title[] % (optional, use only with long paper titles)
{The Aldous--Hoover representation theorem and applications to modeling relational data}

\author % (optional, use only with lots of authors)
{James Lloyd}
% - Use the \inst{?} command only if the authors have different
%   affiliation.

\institute[] % (optional, but mostly needed)
{University of Cambridge}
% - Use the \inst command only if there are several affiliations.
% - Keep it simple, no one is interested in your street address.

\date % (optional)
{January 2013}

\subject{Talks}

\usetikzlibrary{shapes.geometric,arrows,chains,matrix,positioning,scopes}
 \makeatletter
 \tikzset{join/.code=\tikzset{after node path={%
       \ifx\tikzchainprevious\pgfutil@empty\else(\tikzchainprevious)%
       edge[every join]#1(\tikzchaincurrent)\fi}}
 }
 \tikzset{>=stealth',every on chain/.append style={join},
   every join/.style={->}
 }

\tikzstyle{mybox} = [draw=white, rectangle]
\usepackage{ifthen}
\usepackage{booktabs}

% Custom definitions
\def\simiid{\sim_{\mbox{\tiny iid}}}

\input{commenting.tex}

%% For submission, make all render blank.
%\renewcommand{\LATER}[1]{}
%\renewcommand{\fLATER}[1]{}
%\renewcommand{\TBD}[1]{}
%\renewcommand{\fTBD}[1]{}
%\renewcommand{\PROBLEM}[1]{}
%\renewcommand{\fPROBLEM}[1]{}
%\renewcommand{\NA}[1]{#1}  %% Note, NA's pass through!

\begin{document}

\small
%% { 
%%   \setbeamertemplate{footline}{\empty}
%%   \begin{frame}
%%     \titlepage
%%   \end{frame}
%% }
\renewcommand{\inserttotalframenumber}{11}

\input{defs}

\begin{frame}
  \begin{block}{}
    \titlepage
  \end{block}
  %\begin{center}
    %{\bf Thanks to}\\
    %Some people?
  %\end{center}
\end{frame}

\begin{frame}{Agenda}
  \begin{block}{}
  \end{block}
\end{frame}

\begin{frame}{Relational data: definition}
  \begin{block}{}
    Anything measured at pairs or higher of objects.
    
    Anything can be construed into this form, but exchangeability usually tells us if something is sensible
    
    Show a picture of a network
    
    Mention that in full generality we could be talking about a database
  \end{block}
\end{frame}

\begin{frame}{Exchangeability: a weak assumption}
  \begin{block}{}
    Define in mathematics for non relational data
    Explain in words - give an example
  \end{block}
\end{frame}

\begin{frame}{Exchangeability for relational data}
  \begin{block}{}
    Picture of equivalent graphs
  \end{block}
\end{frame}

\begin{frame}{Exchangeable arrays}
  \begin{block}{}
    We need a representation
  
    Pictures of corresponding arrays
    
    Graphs will be working example - but ideas applicable elsewhere
  \end{block}
\end{frame}

\begin{frame}{Exchangeable arrays}
  \begin{block}{}
    We need a representation
  
    Pictures of corresponding arrays
    
    Graphs will be working example - but ideas applicable elsewhere
  \end{block}
\end{frame}

\begin{frame}{Aldous--Hoover theorem}
  \begin{block}{}
  \end{block}
\end{frame}

\begin{frame}{An approximation}
  \begin{block}{}
    State result of Kallenberg
  \end{block}
\end{frame}

\begin{frame}{An approximation: in pictures}
  \begin{block}{}
    Show picture of generating a graph
  \end{block}
\end{frame}

\begin{frame}{Generative model prototype}
  \begin{block}{}
    Write down generic model
    
    Mention how this splits into graph level, node level, edge level
  \end{block}
\end{frame}

\begin{frame}{Many models fit this pattern}
  \begin{block}{}
    Show version that approxes either $W$ or $\Theta$
  \end{block}
\end{frame}

\begin{frame}{Correspondance between GPs and bilinear}
  \begin{block}{}
    State 
  \end{block}
\end{frame}

\begin{frame}{Potential future research - 1-array -> 2-array}
  \begin{block}{}
    \eg Mixture of basis functions (motivate via Mondrian)
    
    Relational $k$-means
    
    Must be something interesting
  \end{block}
\end{frame}

\begin{frame}{Potential future research - 2-array -> 1-array}
  \begin{block}{}
    \eg ILA is an interesting new prior on latent variables
  \end{block}
\end{frame}

\begin{frame}{Potential future research - 1/2-array -> $d$-array}
  \begin{block}{}
    Still of interest but more about computational constraints
  \end{block}
\end{frame}

\begin{frame}{Collections of arrays}
  \begin{block}{}
    What about collections of arrays
    
    Give simple example of feature data
    
    But what if got a social network as well
  \end{block}
\end{frame}

\begin{frame}{Extension of theorem to feature data}
  \begin{block}{}
    Statement
    
    Proof optional
  \end{block}
\end{frame}

\begin{frame}{Extension of theorem to feature data and social network}
  \begin{block}{}
    Just state
  \end{block}
\end{frame}

\begin{frame}{Open questions}
  \begin{block}{}
    Will dropping the dependence between functions be ok?
    
    How do we design models that will not compromise when modeling unrelated arrays
  \end{block}
\end{frame}

\begin{frame}{Bleeding edge thoughts on balancing}
  \begin{block}{}
    Crucial thing is increase in complexity in one array does not pay a penalty based on other arrays and state of model
  \end{block}
\end{frame}

\begin{frame}{Numerical results}
  \begin{block}{}
    Crucial thing is increase in complexity in one array does not pay a penalty based on other arrays and state of model
  \end{block}
\end{frame}

\begin{frame}{Appendix: RFM numerical results}
  \begin{block}{}
    Table
  \end{block}
\end{frame}

\begin{frame}{Appendix: RFM posterior}
  \begin{block}{}
    Pictures
  \end{block}
\end{frame}

\begin{frame}{Appendix: RFM inference}
  \begin{block}{}
    Words and maths
  \end{block}
\end{frame}

\end{document}


