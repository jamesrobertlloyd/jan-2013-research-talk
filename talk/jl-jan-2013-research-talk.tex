\input{../include/header_beamer}

\usecolortheme{default}
\xdefinecolor{Black}{rgb}{0,0,0}
\xdefinecolor{White}{rgb}{1,1,1}
\xdefinecolor{DarkBlue}{rgb}{0,0,.7}
\xdefinecolor{DarkRed}{rgb}{.7,0,0}
\xdefinecolor{Red}{rgb}{.85,0,0}
\xdefinecolor{DarkGreen}{rgb}{0,.7,0}
\xdefinecolor{DarkMagenta}{rgb}{.6,0,.6}
\def\Black{\textcolor{Black}}
\def\White{\textcolor{White}}
\def\Blue{\textcolor{DarkBlue}}
\def\Magenta{\textcolor{DarkMagenta}}
\def\Red{\textcolor{Red}}
\def\Green{\textcolor{DarkGreen}}
\definecolor{camlightblue}{rgb}{0.601 , 0.8, 1}

\usepackage{alltt}
\usepackage{psfrag}
\usepackage{pstool}

\def\newarrow{\mbox{\begin{tikzpicture}
             \useasboundingbox{(-3pt,-4.5pt) rectangle (19pt,1pt)};
             \draw[->] (0,-0.07)--(17pt,-0.07);\end{tikzpicture}}}

\title[] % (optional, use only with long paper titles)
{The Aldous--Hoover representation theorem and applications to modeling relational data}

\author % (optional, use only with lots of authors)
{James Lloyd}
% - Use the \inst{?} command only if the authors have different
%   affiliation.

\institute[] % (optional, but mostly needed)
{University of Cambridge}
% - Use the \inst command only if there are several affiliations.
% - Keep it simple, no one is interested in your street address.

\date % (optional)
{January 2013}

\subject{Talks}

\usetikzlibrary{shapes.geometric,arrows,chains,matrix,positioning,scopes}
 \makeatletter
 \tikzset{join/.code=\tikzset{after node path={%
       \ifx\tikzchainprevious\pgfutil@empty\else(\tikzchainprevious)%
       edge[every join]#1(\tikzchaincurrent)\fi}}
 }
 \tikzset{>=stealth',every on chain/.append style={join},
   every join/.style={->}
 }

\tikzstyle{mybox} = [draw=white, rectangle]
\usepackage{ifthen}
\usepackage{booktabs}

% Custom definitions
\def\simiid{\sim_{\mbox{\tiny iid}}}

\input{../include/commenting.tex}

%% For submission, make all render blank.
%\renewcommand{\LATER}[1]{}
%\renewcommand{\fLATER}[1]{}
%\renewcommand{\TBD}[1]{}
%\renewcommand{\fTBD}[1]{}
%\renewcommand{\PROBLEM}[1]{}
%\renewcommand{\fPROBLEM}[1]{}
%\renewcommand{\NA}[1]{#1}  %% Note, NA's pass through!

%%%%
% Paper specific stuff
%%%%

\newtheorem{thm}{Theorem}%[section]
\newtheorem{lem}[thm]{Lemma}
\newtheorem{prop}[thm]{Proposition}
\newtheorem{cor}[thm]{Corollary}

\newtheorem*{theorem*}{Theorem}

\theoremstyle{definition}
\newtheorem*{definition*}{Definition}
%\newtheorem{definition}[thm]{Definition}%[section]
\newtheorem{conj}{Conjecture}[section]
\newtheorem{exmp}{Example}[section]
\newtheorem{rem}[thm]{Remark}

\theoremstyle{remark}
%\newtheorem{rem}{Remark}
%\newtheorem{note}{Note}
%\newtheorem{case}{Case}

\newcommand{\eqd}{\overset{\,_{\!d}}{=}}
\newcommand{\defn}[1]{\emph{#1}}

\newcommand{\Law}{\mathcal{L}}

\def\given{\,|\,}

\def\SGinf{\mathbb{S}_{\infty}}

\newcommand{\NonNegInts}{\mathbb{Z}_+}
\newcommand{\Nats}{\mathbb{N}}
\newcommand{\Rationals}{\mathbb{Q}}
\newcommand{\Reals}{\mathbb{R}}

\newcommand{\as}{\textrm{a.s.}}

\def\[#1\]{\begin{align}#1\end{align}}
\newcommand{\defas}{:=}

\newcommand{\Normal}{\mathcal{N}}
\newcommand{\dist}{\ \sim\ }

\newcommand{\kernel}{\kappa}
\newcommand{\kernelmatrix}{K}
\newcommand{\scalefactor}{s}
\newcommand{\lengthscale}{\ell}
\newcommand{\targets}{T}
\newcommand{\noise}{\sigma_\targets}
\newcommand{\pseudopoints}{\eta}
\newcommand{\inputpoints}{\xi}
\newcommand{\covhyppar}{\psi}
\newcommand{\logistic}{\phi}

\newcommand{\CompOrder}{\mathcal{O}}
\def\graphspace{\mathbf{G}}
\def\Uniform{\mbox{\rm Uniform}}
\def\Bernoulli{\mbox{\rm Bernoulli}}
\def\ie{i.e.,\ }
\def\eg{e.g.,\ }
\def\iid{i.i.d.\ }
\def\simiid{\sim_{\mbox{\tiny iid}}}
\def\simind{\sim_{\mbox{\tiny ind}}}
\def\eqdist{\stackrel{\mbox{\tiny d}}{=}}
\def\ahfunction{\theta}       
\def\AHfunction{\Theta}           % A-H random function
\def\AHvar{U}                     % A-H uniform variables
\def\AHvaralt{V}                  % A-H uniform variables - for bipartite data
\def\larray{W}                    % latent array sampled with A-H
%\def\latentspace{\mathbf{W}}      % range of entries
\def\latentspace{\mathcal{W}}      % range of entries
\def\darray{X}                    % data array
%\def\dataspace{\mathbf{X}}        % sample space
\def\dataspace{\mathcal{X}}        % sample space
\def\cfspace{\mathbf{C}}          % space of continuous functions
%\def\GP{\mbox{\mathcal{GP}}}
\def\GP{\mathcal{GP}}
\def\likelihood{P}
\def\CovData{C}
\def\CovDataAlt{D}

\begin{document}

\small
%% { 
%%   \setbeamertemplate{footline}{\empty}
%%   \begin{frame}
%%     \titlepage
%%   \end{frame}
%% }
\renewcommand{\inserttotalframenumber}{11}

%\input{../include/defs}

\begin{frame}
  \begin{block}{}
    \titlepage
  \end{block}
  \begin{center}
    {\bf  Collaborators}\\
    Daniel M. Roy (Cambridge)\\
    Peter Orbanz (Columbia)\\
    Zoubin Ghahramani (Cambridge)
  \end{center}
\end{frame}

\begin{frame}{Relational data}
  \begin{block}{}
    Anything measured at more than one `object'
  \end{block}
  \begin{block}{}
  	\center
	\vspace{-1\baselineskip}
    \input{../include/interactome_adj}
  \end{block}
  \begin{block}{}
    In full generality, anything that can be stored in a relational database
  \end{block}
\end{frame}

\begin{frame}{How can we model such data?}
  \begin{block}{}
    \begin{itemize}
      \item Interested in generative modeling of such data for \eg
      \begin{itemize}
        \item Discovery of latent structure \eg groups of proteins with similar functions in protein-protein interactomes
        \item Prediction of missing data \eg movie recommendation, friend suggestions
      \end{itemize}
      \vspace{\baselineskip}
      \item Relational data typically encoded in arrays. How do reasonable assumptions about the data translate to the array representation?
      \vspace{\baselineskip}
      \item We make a weak assumption and use a result from probability theory to describe the implied structure of arrays
      \begin{itemize}
        \item Implied structure allows for classification of many models
        \item Also inspires a simple Bayesian nonparametric model with good empirical performance
      \end{itemize}
    \end{itemize}
  \end{block}
\end{frame}

\begin{frame}{Exchangeability for relational data}
  \begin{block}{}
  \center
  \input{../include/exchangeable_just_graphs}
  \end{block}
\end{frame}

\begin{frame}{Exchangeability for corresponding arrays}
  \begin{block}{}
  \center
  \input{../include/exchangeable}
  \end{block}
\end{frame}

\begin{frame}{Exchangeability can be characterised}
\begin{block}{}
  \begin{definition*}
  An array $\darray=(\darray_{ij})_{i,j\in\Nats}$ is called an \emph{exchangeable array} if 
  \begin{equation*}
    \label{eq:jointly:ex}
    (\darray_{ij})\eqdist(\darray_{\pi(i)\pi(j)}) \qquad\text{ for every }\pi\in\SGinf\;.
  \end{equation*}
\end{definition*}
\end{block}

%\vspace{1\baselineskip}
\begin{block}{}
\begin{theorem*}[Aldous, Hoover]
  \label{theorem:ah}
  A random 2-array $(\darray_{ij})$ is exchangeable if and only if there is a random (measurable) function ${F:[0,1]^3\rightarrow\dataspace}$ such that 
  \begin{equation*}
    \label{eq:ah}
    (\darray_{ij})\eqdist (F(\AHvar_i,\AHvar_j,\AHvar_{ij})).
  \end{equation*}
for every collection  $(\AHvar_i)_{i\in\Nats}$ and $(\AHvar_{ij})_{i\le j\in\Nats}$ of \iid $\Uniform[0,1]$ random variables, where $\AHvar_{ji} = \AHvar_{ij}$ for $j < i \in \Nats$.
\end{theorem*}
\end{block}
\end{frame}

\begin{frame}{An arbitrarily good approximation}
  \begin{block}{This representation can be simplified}
Call an array $(\darray_{ij})$, \emph{simple} if it admits a representation
\[
(\darray_{ij}) \eqd (\Theta(\AHvar_{i},\AHvar_{j})) \nonumber
\]
Let $\Law(Y)$ be the law (distribution) of a random variable $Y$ and \\define $\chi_m X \defas (X_{ij}; \ i,j \le m )$.
\end{block}
\begin{block}{}
\begin{thm}[{Kallenberg}]
\label{theorem:simple}
Let $\darray$ be a $d$-dimensional exchangeable array in a Borel space $\dataspace$.  Then there exist some simple exchangeable arrays $X_1,X_2,\dotsc$ such that $\Law(\chi_m X_n) \textrm{ and } \Law(\chi_m X)$ are mutually absolutely continuous for all $m,n \in \Nats$ and the associated Radon--Nikodym derivatives converge uniformly to 1 as $n \to \infty$ for fixed $m$.
\end{thm}
  \end{block}
%  \begin{block}{\ldots allowing the following simplification}
%    We can produce arbitrarily good approximations to exchangeable arrays with functions of the form
%    \[
%      (\darray_{ij}) = (F(\AHvar_i,\AHvar_j)) \nonumber
%    \]
%  \end{block}
\end{frame}

\begin{frame}{This inspires a Bayesian nonparametric model}
\begin{block}{}
We decompose the function $F$ into two functions 
${\AHfunction:[0,1]^2\rightarrow\latentspace}$ 
and ${H:[0,1]\times\latentspace\rightarrow\dataspace}$ for a suitable space $\latentspace$\/, such that
\begin{equation*}
  \label{eq:decomposition}
  (\darray_{ij}) \eqd (F(\AHvar_i,\AHvar_j,\AHvar_{ij}))=(H(\AHvar_{ij},\AHfunction(\AHvar_i,\AHvar_j)))\;.
\end{equation*}
\end{block}
\begin{block}{}
Inspiring the following generative model
\begin{equation}
  \begin{split}
    \AHfunction\; &\sim\; \GP(0,\kernel) \\
    \AHvar_{1},\AHvar_{2},\ldots\; &\simiid\; \Uniform[0,1] \nonumber \\
    \darray_{ij}\,|W_{ij}\; &\sim\; \likelihood[\,.\,|W_{ij}]
  \end{split}
\end{equation}

\begin{center}
where $W_{ij}=\AHfunction(\AHvar_i,\AHvar_j)$.
\end{center}
\end{block}
\end{frame}

\begin{frame}{The model  in pictures}
 \begin{block}{}
   \input{../include/fig_graphon.tex}
 \end{block}
\end{frame}

\begin{frame}{Many models fit this pattern}
  \begin{block}{}
\begin{center}
  \begin{tabular}{lccl} 
%    \toprule
    \multicolumn{4}{c}{Graph data}\\
    \midrule
    %\addlinespace[2pt]
    %\textcolor{red}{Random function model} 
    Random function model & $\AHfunction$ & $\sim$ & $\GP\,(0, \kernel)$\\
    Latent class & $\larray_{ij}$ & $=$ & $\Lambda_{\AHvar_i\AHvar_j}\,\textrm{where} \,\AHvar_i \in \{1,\ldots,K\}$\\
    IRM & $\larray_{ij}$ & $=$ & $\Lambda_{\AHvar_i\AHvar_j}\,\textrm{where} \,\AHvar_i \in \{1,\ldots,\infty\}$\\
    Latent distance & $\larray_{ij}$ & $=$ & $-|\AHvar_i - \AHvar_j|$\\
    Eigenmodel &$\larray_{ij}$ & $=$ & $\AHvar_i'\Lambda \AHvar_j$\\
    LFRM & $\larray_{ij}$ & $=$ & $\AHvar_i'\Lambda \AHvar_j\,\textrm{where} \,\AHvar_i \in \{0,1\}^\infty$\\
    ILA & $\larray_{ij}$ & $=$ & $\sum_d \mathbb{I}_{U_{id}}\mathbb{I}_{U_{jd}}\Lambda^{(d)}_{U_{id}U_{jd}}\,\textrm{where} \,\AHvar_i \in \{0,\ldots,\infty\}^\infty$\\
    SMGB & $\AHfunction$ & $\dist$ & $\GP\,(0, \kernel_1 \otimes \kernel_2)$ \\
    %\midrule
    \addlinespace[4pt]
    \multicolumn{4}{c}{Real-valued array data}\\
    \midrule
    %\textcolor{red}{Random function model} 
    Random function model & $\AHfunction$ & $\sim$ & $\GP\,(0, \kernel)$\\
    Mondrian process based & $\AHfunction$ & = & piece-wise constant random function\\
    PMF & $\larray_{ij}$ & $=$ & $\AHvar_i'V_j$\\
    GPLVM & $\AHfunction$ & $\sim$ & $\GP\,(0, \kernel \otimes \delta)$\\
    %Linear relational GP~\cite{Yu2008} & $\AHfunction$ & $\sim$ & $\GP\,(0, \kernel_1 \otimes \kernel_2)\,\textrm{where} \,\kernel_i\,\textrm{is linear} $\\
%\bottomrule
\end{tabular}
\end{center}
  \end{block}
\end{frame}

\begin{frame}{A correspondence result}
  \begin{block}{}
\begin{prop}
\label{prop:matrixfactorisation}
A matrix factorization model defined as
\begin{equation*}
W_{ij} = \AHvar_i'\Lambda\AHvaralt_j \quad \quad \Lambda_{ij} \simiid \Normal (0, 1)
\end{equation*}
is equivalent to
\begin{equation*}
W_{ij} = \AHfunction\,(\AHvar_i, \AHvaralt_j) \quad \quad \AHfunction \sim \GP\,(0, L_\AHvar \otimes L_\AHvaralt)
\end{equation*}
\vspace{\baselineskip}

where $L_U(\AHvar_{i_1}, \AHvar_{i_2}) = \AHvar_{i_1}'\AHvar_{i_2}$ and similarly for $L_V$.
\end{prop}
  \end{block}
\end{frame}

\begin{frame}{A simpler overview of model space}
  \begin{block}{}
  \centering
  \begin{tabular}{l|ccc}%c} 
%    \toprule
    & $W_{ij}$ & $\kernel$ & $U_i, V_j \sim \, .$ \\% & $\dataspace$\\ 
    %\multicolumn{4}{c}{Graph data}\\
    \midrule
    %\addlinespace[2pt]
    %\textcolor{red}{Random function model} 
    Random function model & $\phi(U_i, V_j)'\Lambda$ & $\kernel_{U\times V}$ & Gaussian\\% & All\\
    SMGB, InfTucker & $\phi(\AHvar_i)'\Lambda\phi(\AHvaralt_j)$ & $\kernel_U \otimes \kernel_V$ & Laplace \\% & $\{0,1\},\, \Reals$\\
    GPLVM & $\phi(\AHvar_i)'\Lambda$ & $\kernel_\AHvar \otimes \delta_V$ & Gaussian \\% & All\\
    Eigenmodel & $\AHvar_i'\Lambda \AHvaralt_j$ & $L_U \otimes L_V$ & Gaussian \\% & All\\
    Linear relational GP & $\AHvar_i'\Lambda \AHvaralt_j$ & $L_U \otimes L_V$ & Gaussian \\% & $\{0,1\}$\\
    PCA, PMF & $\AHvar_i'\Lambda$ & $L_U \otimes \delta_V$ & Gaussian \\% & $\Reals$\\
    Latent distance & $-|\AHvar_i - \AHvar_j|$ & 0 & Gaussian \\% & $\{0,1\}$\\
    Mondrian process based & Decision tree & * & Uniform \\% & All\\
    \midrule
    Latent class & $\Lambda_{\AHvar_i\AHvar_j}$ & $\delta_{U\times U}$ & Multinomial \\% & $\{0,1\}$\\
    IRM &$\Lambda_{\AHvar_i\AHvaralt_j}$ & $\delta_{U\times V}$ & CRP \\% & $\{0,1\}$\\
    IHRM  &$\Lambda_{\AHvar_i\AHvaralt_j}$ & $\delta_{U\times V}$ & CRP \\% & $\Reals$\\
    BMF  & $\AHvar_i'\Lambda \AHvaralt_j$ & $L_U \otimes L_V$ & IBP \\% & $\Reals$\\
    LFRM & $\AHvar_i'\Lambda \AHvar_j$ & $L_U \otimes L_U$ & IBP \\% & $\{0,1\}$\\
    ILA & $\sum_d \mathbb{I}_{U_{id}}\mathbb{I}_{U_{jd}}\Lambda^{(d)}_{U_{id}U_{jd}}$ & * & CRP $+$ IBP \\% &$\{0,1\}$ \\
    %\midrule
    %\addlinespace[4pt]
    %\multicolumn{4}{c}{Real-valued matrix data}\\
    %\midrule
    %\textcolor{red}{Random function model} 
%\bottomrule
\end{tabular}
  \end{block}
\end{frame}

\begin{frame}{Posterior}
  \begin{block}{}
  \begin{center}
  \begin{tikzpicture}[scale=3]
  \begin{scope}[xshift=0cm]
    \node [mybox] (box){
      \includegraphics[width=0.3\textwidth]{../figures/graph_standard.pdf}
    };
    \node[inner sep=0,text width=0.3\textwidth, text centered] (note1) at (0,-0.06\textwidth) {
         A protein interactome};
  \end{scope}
  \begin{scope}[xshift=0.12\textwidth]
    \node [mybox] (box){
      \includegraphics[width=0.24\textwidth]{../figures/interactome_adjacency_mathematica.pdf}
  };
    \node[inner sep=0,text width=0.3\textwidth, text centered] (note1) at (0,-0.06\textwidth) {
         Adjacency matrix sorted by MAP embedding};
  \end{scope}  
  \begin{scope}[xshift=0.24\textwidth]
    \node [mybox] (box){
      \includegraphics[width=0.3\textwidth]{../figures/graphon_listplot_bmp.pdf}
  };
    \node[inner sep=0,text width=0.3\textwidth, text centered] (note1) at (0,-0.06\textwidth) {
         MAP $\Theta$};
  \end{scope}
\end{tikzpicture}

\end{center}
  \end{block}
\end{frame}

\begin{frame}{Posterior}
  \begin{block}{}
  Show video if available
  
  Discuss block models, stochastic equivalence and homophily
  \end{block}
\end{frame}

\begin{frame}{Ongoing research / ideas}
  \begin{block}{}
    \begin{itemize}
      \item Modeling multiple arrays \eg joint modelling of social network and `like' data
      \begin{itemize}
        \item Corollaries of Aldous--Hoover suggest representations for such data
        \item Many unanswered questions about generating good models
      \end{itemize}
      \vspace{\baselineskip}
      \item Trying new priors on functions
      \begin{itemize}
        \item Many priors on functions for sequential data that could have utility for relational data
        \item \eg Simpler basis function representation of random functions?
      \end{itemize}
      \vspace{\baselineskip}
      \item Trying new priors on latent variables
      \begin{itemize}
        \item CRP $+$ IBP prior in ILA could be more broadly applicable \eg sequential data
      \end{itemize}
    \end{itemize}
  \end{block}
\end{frame}

\begin{frame}{Extensions: Array with `feature' data}
  \begin{block}{}
\begin{cor}
  \label{theorem:ahcov}
  Let $(\darray_{ij})_{i,j\in\Nats}$ and $(\CovData_i)_{i\in\Nats}$ be random variables in $\dataspace$ and $\dataspace'$ respectively.  Then the following are equivalent:
\begin{enumerate}  
\item[i.] $(\darray_{ij}, \CovData_i)\eqdist(\darray_{\pi(i)\pi(j)}, \CovData_{\pi(i)}) \,\text{ for every }\pi\in\SGinf$. 
\item[ii.] There are random (measurable) functions $F:[0,1]^3\rightarrow\dataspace$ and $G:[0,1]\rightarrow\dataspace'$ such that 
   \begin{equation}
    \label{eq:ah:cov}
    (\darray_{ij}, \CovData_i)\eqdist (F(\AHvar_i,\AHvar_j,\AHvar_{ij}), G(\AHvar_i)), \nonumber
  \end{equation}
for every collection $(\AHvar_i)_{i\in\Nats}$ and $(\AHvar_{ij})_{i\le j\in\Nats}$ of \iid $\Uniform[0,1]$ random variables, where $\AHvar_{ji} = \AHvar_{ij}$ for $j < i \in \Nats$.
\end{enumerate}
\end{cor}
  \end{block}
\end{frame}

\begin{frame}{Extensions: Multiple arrays}
  \begin{block}{}
Consider rating data $(\darray_{ij})$ with users $i$ and movies $j$, with side information in the form of covariates for both users, $\CovData_i$, and movies, $\CovDataAlt_j$, and a social network $(S_{ik})$ over users $i,k$.
\end{block}
  \begin{block}{}
\begin{cor}
The following are equivalent
\begin{enumerate}
\item[i.] $
(\darray_{ij}, \CovData_i, \CovDataAlt_j, S_{ik}) \eqdist
(\darray_{\pi(i)\pi'(j)}, \CovData_{\pi(i)}, \CovDataAlt_{\pi'(j)}, S_{\pi(i)\pi(k)})$ 
for every $\pi,\pi' \in \SGinf$.
\item[ii.] There exist random functions $F,G,H,I$ such that
\[
(\darray_{ij}, \CovData_i, \CovDataAlt_j, S_{ik}) \eqdist (F(\AHvar_i,\AHvaralt_j,W_{ij}), G(\AHvar_i), H(\AHvaralt_j), I(\AHvar_i,\AHvar_k,\AHvar_{ik}))\nonumber
\]
for every collection $(\AHvar_i)_{i\in\Nats}, (\AHvaralt_j)_{j\in\Nats}, (W_{ij})_{i, j\in\Nats}$ and $(\AHvar_{ik})_{i\le k\in\Nats}$ of \iid $\Uniform[0,1]$ random variables, where $\AHvar_{ki} = \AHvar_{ik}$ for $k < i \in \Nats$.
\end{enumerate}
\end{cor}
  \end{block}
\end{frame}

\begin{frame}{Multiple arrays: preliminary results}
 \begin{block}{Data}
   \begin{itemize}
   \item A friend of friends network collected from last.FM ($S_{ik}$)
   \item A user $\times$ genre matrix: $X_{ij} = 1$ iff user $i$ has listened to genre $j$
   \end{itemize}
 \end{block}
 \begin{block}{Cold start task}
   \begin{itemize}
     \item Want to predict entire rows of $X_{ij}$ \ie recommendations for new users
     \item Consider jointly modelling the array
   \end{itemize}
 \end{block}
 \begin{block}{Preliminary numerical results promising}
   \begin{itemize}
     \item Jointly modelling social network raises AUC from 0.50 to 0.55
     \item Improvement over heuristics \eg population average, average of friends, \ldots
     \item More research required to understand significance of result
   \end{itemize} 
 \end{block}
\end{frame}

\begin{frame}{Multiple arrays: open questions}
  \begin{block}{}
    \begin{itemize}
      \item Which designs of model will effectively model multiple arrays without having to `balance' or compromise?
      \begin{itemize}
        \item Flat clustering models seem especially inappropriate \eg IRM
        \item Multiple clustering models seem well suited \eg IBP priors
        \item How does this transfer to GP case - in particular, prior on length scales?
        \item Can good design principles identify an expressive yet tractable model that will provide an effective tool for cold start problems
      \end{itemize}
      \vspace{\baselineskip}
      \item Is generative modelling appropriate, or can we find more efficient models of conditional densities?
      \begin{itemize}
        \item What are appropriate representations for conditional densities?
      \end{itemize}
      \end{itemize}
  \end{block}
\end{frame}

\begin{frame}{Other priors on functions}
  \begin{block}{}
    Draw IRM, Mondrian, mixture of Gaussians
  \end{block}
\end{frame}

\begin{frame}{Appendix: RFM numerical results}
  \begin{block}{}
    \begin{center}
    \tiny
  \begin{tabular}{r | r r r | r r r | r r r}
    \multicolumn{10}{c}{AUC results} \\
    \addlinespace[2pt]
    Data set & \multicolumn{3}{c|}{High school} & \multicolumn{3}{c|}{NIPS} & \multicolumn{3}{c}{Protein} \\
    Latent dim. & 1 & 2 & 3 & 1 & 2 & 3 & 1 & 2 & 3 \\
    \midrule
    PMF                   & 0.747 & 0.792 & 0.792 & 0.729 & 0.789 & 0.820 & 0.787 & 0.810 & 0.841 \\
    Eigenmodel            & 0.742 & 0.806 & 0.806 & 0.789 & 0.818 & 0.845 & 0.805 & 0.866 & 0.882 \\
    GPLVM                 & 0.744 & 0.775 & 0.782 & 0.888 & 0.876 & 0.883 & 0.877 & 0.883 & 0.873 \\
    RFM & \textbf{0.815} & \textbf{0.827} & \textbf{0.820} & \textbf{0.907} & \textbf{0.914} & \textbf{0.919} & \textbf{0.903} & \textbf{0.910} & \textbf{0.912}
  \end{tabular}
\end{center}
  \end{block}
\end{frame}

\begin{frame}{Appendix: RFM inference}
  \begin{block}{}
\begin{itemize}
\item Inducing points in subset of regressors appoximation treated as random variables (locations and values)
\vspace{2\baselineskip}
\item Gaussian process sampled using elliptical slice sampling; all other parameters sampled using slice sampling
\end{itemize}
  \end{block}
\end{frame}

\begin{frame}{Appendix: proof sketch of corollary}
\begin{proof}
By Aldous--Hoover, there exists a random function $F'$ such that
\[
(R_{ij}) \eqdist (F'(\AHvar_i,\AHvar_j,\AHvar_{ij})), \nonumber
\]
and so we may write 
\[
(\darray_{ij}, \CovData_i) \eqdist 
     (F_1'(\AHvar_i,\AHvar_j,\AHvar_{ij}), F_2'(\AHvar_i,\AHvar_j,\AHvar_{ij})) \nonumber
\]
for a pair of random functions $F'_1$ and $F'_2$ into $\dataspace$ and $\dataspace'$, respectively.
It follows that, with probability one,
$F_2'(\AHvar_i,\AHvar_j,\AHvar_{ij}) = F_2'(\AHvar_i,\AHvar_i,\AHvar_{ii})$ for all $i,j\in\Nats$
and so, in particular we have 
\[
(\darray_{ij}, \CovData_i) \eqdist 
       (F_1'(\AHvar_i,\AHvar_j,\AHvar_{ij}),F_2'(\AHvar_i,\AHvar_i,\AHvar_{ii})) \nonumber
\]
Let $\phi: [0,1]^2 \to [0,1]$ be a bimeasurable function such that
$U_i^* = \phi(\AHvar_i,\AHvar_{ii})$ is uniformly distributed.
Then we can write
\[
(\darray_{ij}, \CovData_i) \eqdist (F(\AHvar_i^*,\AHvar_j^*,\AHvar_{ij}), G(\AHvar_i^*)), \nonumber
\]
where $F$ and $G$ are given by $F(\phi(x,xx),\phi(y,yy),z) = F_1'(x,y,z)$ and $G(\phi(x,xx)) = F_2'(x,x,xx)$, completing the proof.
\end{proof}
\end{frame}

\end{document}


